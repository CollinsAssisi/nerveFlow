% Template for PLoS
% Version 3.5 March 2018
%
% % % % % % % % % % % % % % % % % % % % % %
%
% -- IMPORTANT NOTE
%
% This template contains comments intended 
% to minimize problems and delays during our production 
% process. Please follow the template instructions
% whenever possible.
%
% % % % % % % % % % % % % % % % % % % % % % % 
%
% Once your paper is accepted for publication, 
% PLEASE REMOVE ALL TRACKED CHANGES in this file 
% and leave only the final text of your manuscript. 
% PLOS recommends the use of latexdiff to track changes during review, as this will help to maintain a clean tex file.
% Visit https://www.ctan.org/pkg/latexdiff?lang=en for info or contact us at latex@plos.org.
%
%
% There are no restrictions on package use within the LaTeX files except that 
% no packages listed in the template may be deleted.
%
% Please do not include colors or graphics in the text.
%
% The manuscript LaTeX source should be contained within a single file (do not use \input, \externaldocument, or similar commands).
%
% % % % % % % % % % % % % % % % % % % % % % %
%
% -- FIGURES AND TABLES
%
% Please include tables/figure captions directly after the paragraph where they are first cited in the text.
%
% DO NOT INCLUDE GRAPHICS IN YOUR MANUSCRIPT
% - Figures should be uploaded separately from your manuscript file. 
% - Figures generated using LaTeX should be extracted and removed from the PDF before submission. 
% - Figures containing multiple panels/subfigures must be combined into one image file before submission.
% For figure citations, please use "Fig" instead of "Figure".
% See http://journals.plos.org/plosone/s/figures for PLOS figure guidelines.
%
% Tables should be cell-based and may not contain:
% - spacing/line breaks within cells to alter layout or alignment
% - do not nest tabular environments (no tabular environments within tabular environments)
% - no graphics or colored text (cell background color/shading OK)
% See http://journals.plos.org/plosone/s/tables for table guidelines.
%
% For tables that exceed the width of the text column, use the adjustwidth environment as illustrated in the example table in text below.
%
% % % % % % % % % % % % % % % % % % % % % % % %
%
% -- EQUATIONS, MATH SYMBOLS, SUBSCRIPTS, AND SUPERSCRIPTS
%
% IMPORTANT
% Below are a few tips to help format your equations and other special characters according to our specifications. For more tips to help reduce the possibility of formatting errors during conversion, please see our LaTeX guidelines at http://journals.plos.org/plosone/s/latex
%
% For inline equations, please be sure to include all portions of an equation in the math environment.  For example, x$^2$ is incorrect; this should be formatted as $x^2$ (or $\mathrm{x}^2$ if the romanized font is desired).
%
% Do not include text that is not math in the math environment. For example, CO2 should be written as CO\textsubscript{2} instead of CO$_2$.
%
% Please add line breaks to long display equations when possible in order to fit size of the column. 
%
% For inline equations, please do not include punctuation (commas, etc) within the math environment unless this is part of the equation.
%
% When adding superscript or subscripts outside of brackets/braces, please group using {}.  For example, change "[U(D,E,\gamma)]^2" to "{[U(D,E,\gamma)]}^2". 
%
% Do not use \cal for caligraphic font.  Instead, use \mathcal{}
%
% % % % % % % % % % % % % % % % % % % % % % % % 
%
% Please contact latex@plos.org with any questions.
%
% % % % % % % % % % % % % % % % % % % % % % % %

\documentclass[10pt,letterpaper]{article}
\usepackage[top=0.85in,left=2.75in,footskip=0.75in]{geometry}

% amsmath and amssymb packages, useful for mathematical formulas and symbols
\usepackage{amsmath,amssymb}

% Use adjustwidth environment to exceed column width (see example table in text)
\usepackage{changepage}

% Use Unicode characters when possible
\usepackage[utf8x]{inputenc}

% textcomp package and marvosym package for additional characters
\usepackage{textcomp,marvosym}

% cite package, to clean up citations in the main text. Do not remove.
\usepackage{cite}

% Use nameref to cite supporting information files (see Supporting Information section for more info)
\usepackage{nameref,hyperref}

% line numbers
\usepackage[right]{lineno}

% ligatures disabled
\usepackage{microtype}
\DisableLigatures[f]{encoding = *, family = * }

% color can be used to apply background shading to table cells only
\usepackage[table]{xcolor}

% array package and thick rules for tables
\usepackage{array}

% code-listing package
\usepackage{minted}

% create "+" rule type for thick vertical lines
\newcolumntype{+}{!{\vrule width 2pt}}

% create \thickcline for thick horizontal lines of variable length
\newlength\savedwidth
\newcommand\thickcline[1]{%
  \noalign{\global\savedwidth\arrayrulewidth\global\arrayrulewidth 2pt}%
  \cline{#1}%
  \noalign{\vskip\arrayrulewidth}%
  \noalign{\global\arrayrulewidth\savedwidth}%
}

% \thickhline command for thick horizontal lines that span the table
\newcommand\thickhline{\noalign{\global\savedwidth\arrayrulewidth\global\arrayrulewidth 2pt}%
\hline
\noalign{\global\arrayrulewidth\savedwidth}}


% Remove comment for double spacing
%\usepackage{setspace} 
%\doublespacing

% Text layout
\raggedright
\setlength{\parindent}{0.5cm}
\textwidth 5.25in 
\textheight 8.75in

% Bold the 'Figure #' in the caption and separate it from the title/caption with a period
% Captions will be left justified
\usepackage[aboveskip=1pt,labelfont=bf,labelsep=period,justification=raggedright,singlelinecheck=off]{caption}
\renewcommand{\figurename}{Fig}

% Use the PLoS provided BiBTeX style
\bibliographystyle{plos2015}

% Remove brackets from numbering in List of References
\makeatletter
\renewcommand{\@biblabel}[1]{\quad#1.}
\makeatother



% Header and Footer with logo
\usepackage{lastpage,fancyhdr,graphicx}
\usepackage{epstopdf}
%\pagestyle{myheadings}
\pagestyle{fancy}
\fancyhf{}
%\setlength{\headheight}{27.023pt}
%\lhead{\includegraphics[width=2.0in]{PLOS-submission.eps}}
\rfoot{\thepage/\pageref{LastPage}}
\renewcommand{\headrulewidth}{0pt}
\renewcommand{\footrule}{\hrule height 2pt \vspace{2mm}}
\fancyheadoffset[L]{2.25in}
\fancyfootoffset[L]{2.25in}
\lfoot{\today}

%% Include all macros below

\newcommand{\lorem}{{\bf LOREM}}
\newcommand{\ipsum}{{\bf IPSUM}}

%% END MACROS SECTION


\begin{document}
\vspace*{0.2in}

% Title must be 250 characters or less.
\begin{flushleft}
{\Large
\textbf\newline{Parallelised Scalable Simulations of Biological Neural Networks using TensorFlow: A Beginners’ Guide} % Please use "sentence case" for title and headings (capitalize only the first word in a title (or heading), the first word in a subtitle (or subheading), and any proper nouns).
}
\newline
% Insert author names, affiliations and corresponding author email (do not include titles, positions, or degrees).
\\
Saptarshi Soham Mohanta\textsuperscript{1},
Collins Assisi\textsuperscript{1},
with Theoretical Neuroscience Group\textsuperscript{\textpilcrow}.
\\
\bigskip
\textbf{1} Indian Institute of Science Education and Research, Pune, Maharashtra, India
\\
\bigskip


% Current address notes
\textcurrency Current Address: Biology Department, Indian Institute of Science Education and Research, Pune, Maharashtra, India 

% Group/Consortium Author Note
\textpilcrow Membership list can be found in the Acknowledgments section.

% Use the asterisk to denote corresponding authorship and provide email address in note below.
* collins@iiserpune.ac.in

\end{flushleft}
% Please keep the abstract below 300 words
\section*{Abstract}
The dynamics of neurons and their networks have been studied extensively by modelling them as collections of differential equations, and the power of these mathematical tools are well recognised. Many tools and packages exist that allow the simulation of systems of neurons using these differential equations. However, there is a barrier of entry in terms of developing flexible general purpose simulations that are platform independent and support hardware acceleration on modern computing architectures such as GPUs/TPUs and Distributed Platforms. TensorFlow is a Python-based open-source package initially designed for machine learning algorithms, but it presents a scalable environment for a variety of computation including solving differential equations using iterative algorithms from numerical analysis such as Runge-Kutta methods. There are two significant benefits of such an implementation: high readability and scalability across a variety of computational devices. We explore the process of implementing a scalable simulation of a system of neurons based on Hodgkin-Huxley-like neuron equations using TensorFlow. We also discuss the limitations of such implementation and approaches to deal with them. 


% Please keep the Author Summary between 150 and 200 words
% Use first person. PLOS ONE authors please skip this step. 
% Author Summary not valid for PLOS ONE submissions.   
\section*{Author summary}
In the form of a 7-day tutorial, the reader is introduced to the mathematical modelling of neuronal networks based on the Hodgkin-Huxley Differential Equations and is instructed on developing highly parallelised but easily readable code for numerical methods such as Euler's Method and Runge-Kutta Methods to solve differential equations in Python and use them to simulate neuronal dynamics. To develop scalable code, Google's open-source package TensorFlow is introduced, and the reader is instructed in developing simulations using this package and handling the few limitations that come with this implementation. The reader is also introduced to coding structures that maximise the parallelizability of the simulation. Finally, the coding paradigm that was developed is used to simulate a model of Locus Antennal Lobe described in previous literature, and its efficacy is analysed. 

\linenumbers

% Use "Eq" instead of "Equation" for equation citations.
\section*{Introduction}
The processing of information by the nervous system spans across space and time, and mathematical modelling of these dynamics have found to be an essential tool. These models have been used extensively to study the dynamics and mechanisms of information processing at both the individual neuron level and the system of neurons level. These models generally utilise systems of simultaneous ordinary differential equations (ODEs) which are solved as initial value problems using well-studied methods from numerical analysis such as Euler's Method and Runge Kutta methods. From a detailed study of the mechanism of action of the neurons, ion channels, neurotransmitters or neuromodulators and their dynamics in different models, equations have been found that describe the behaviour of neurons and synapse. By forming interconnected systems of these individual groups of differential equations, the overall dynamics and behaviour of networks can be studied through deterministic or stochastic simulations which can be easily perturbed unlike the case for in vivo experiments.

A significant issue with such simulations is computational complexity. As the number of neurons increase, the number of possible synaptic connections increases quadratically. That is, for a system of $n$ neurons there can be at most $n^2$ different synapses of one type, each with its own set of equations. Thus, simulations can take very long times for large values of $n$. A solution to this problem is to implement some form of parallelisation in the ODE solver and the system of equations itself. One of the simplest methods of parallelizable computation is in the form of matrix algebra which can be accelerated using libraries such as BLAS which can only be used for accelerating CPU based computations. Similarly, CUDA is available for speeding up computations on Nvidia-based GPUs and TPUs. However, there is a barrier of entry to using low-level packages like CUDA for the general user as it sometimes requires an in-depth understanding of the architecture, particularly for troubleshooting.

This is where TensorFlow (an open-source Google product) gives us a massive edge. TensorFlow allows us much greater scalability and is way more flexible in terms of ease of implementation for specialised hardware. With minimal changes in the implementation, the code can be executed on a wide variety of heterogeneous and distributed systems ranging from mobile devices to clusters and specialised computing devices such as GPU and TPU cards. The modern advances in GPU/TPU design allow us to access even higher degrees of parallelisation. It is now even possible to have hundred of TeraFLOPS of computing power in a single small computing device. With TensorFlow, we can access these resources without even requiring an in-depth understanding of its architecture and technical knowledge of specific low-level packages like CUDA.

%Lorem ipsum dolor sit~\cite{bib1} amet, consectetur adipiscing elit. Curabitur eget porta erat. Morbi consectetur est vel gravida pretium. Suspendisse ut dui eu ante cursus gravida non sed sem. Nullam Eq~(\ref{eq:schemeP}) sapien tellus, commodo id velit id, eleifend volutpat quam. Phasellus mauris velit, dapibus finibus elementum vel, pulvinar non tellus. Nunc pellentesque pretium diam, quis maximus dolor faucibus id.~\cite{bib2} Nunc convallis sodales ante, ut ullamcorper est egestas vitae. Nam sit amet enim ultrices, ultrices elit pulvinar, volutpat risus.

%\begin{eqnarray}
%\label{eq:schemeP}
%	\mathrm{P_Y} = \underbrace{H(Y_n) - H(Y_n|\mathbf{V}^{Y}_{n})}_{S_Y} + \underbrace{H(Y_n|\mathbf{V}^{Y}_{n})- H(Y_n|\mathbf{V}^{X,Y}_{n})}_{T_{X\rightarrow Y}},
%\end{eqnarray}

\section*{Materials and methods}
\subsection*{Requirements for the Tutorial}

For this tutorial, the reader is expected to have an understanding of Python and some commonly used packages such as Numpy and Matplotlib. The reader is also expected to know some amount of calculus particularly the theory of differential equations. Access to a computer will the following prerequisite software/packages is preferable: Python 3.6 or above, Jupyter Notebook, Numpy Python package, Matplotlib Python package, and TensorFlow 1.13 or above. All software can be installed using Anaconda Distribution of Python 3. Instructions for TensorFlow installation is available on their website.

% For figure citations, please use "Fig" instead of "Figure".
%Nulla mi mi, Fig~\ref{fig1} venenatis sed ipsum varius, volutpat euismod diam. Proin rutrum vel massa non gravida. Quisque tempor sem et dignissim rutrum. Lorem ipsum dolor sit amet, consectetur adipiscing elit. Morbi at justo vitae nulla elementum commodo eu id massa. In vitae diam ac augue semper tincidunt eu ut eros. Fusce fringilla erat porttitor lectus cursus, \nameref{S1_Video} vel sagittis arcu lobortis. Aliquam in enim semper, aliquam massa id, cursus neque. Praesent faucibus semper libero.

% Place figure captions after the first paragraph in which they are cited.
%\begin{figure}[!h]
%\caption{{\bf Bold the figure title.}
%Figure caption text here, please use this space for the figure panel descriptions instead of using subfigure commands. A: Lorem ipsum dolor sit amet. B: Consectetur adipiscing elit.}
%\label{fig1}
%\end{figure}

\subsection*{Day 1: Of Numerical Integration, Python and Tensorflow}

Our discussion begins with what Numerical Integration is and how we can use it to solve differential equations given the initial condition in  Python using Numpy or TensorFlow.

\subsubsection*{What is Numerical Integration?}

For a theoretician, the ideal form of the solution to a differential equation given the initial conditions, i.e. an initial value problem (IVP), would be a formula for the solution function. However, at times obtaining a formulaic solution is not easy, and in many cases it is impossible. So, what do we do when faced with a differential equation that we cannot solve? If one is only looking for long term behaviour of a solution, one can always sketch a direction field. This can be done without too much difficulty for some reasonably complex differential equations that we cannot solve to get exact solutions. However, what if we need to determine how a specific solution behaves, including some values that the solution will take? In that case, we have to rely on numerical methods for solving the IVP such as Euler's method or the Runge-Kutta Methods. 

\subsubsection*{Euler's Method}

We use Euler's Method to generate a numerical solution to an initial value problem of the form:

\begin{eqnarray}
	\frac{dx}{dt} = f(x, t)
\end{eqnarray}
\begin{eqnarray}
	x(t_o) = x_o
\end{eqnarray}

Firstly, we decide the interval over which we desire to find the solution, starting at the initial condition. We break this interval into small subdivisions of a fixed length $\epsilon$. Then, using the initial condition as our starting point, we generate the rest of the solution by using the iterative formulas:

\begin{eqnarray}
	t_{n+1} = t_n + \epsilon
\end{eqnarray}
\begin{eqnarray}
	x_{n+1} = x_n + \epsilon f(x_n, t_n)
\end{eqnarray}

to find the coordinates of the points in our numerical solution. We end this process once we have reached the end of the desired interval.

\subsubsection*{Euler Method in Python}
Let $\frac{dx}{dt}=f(x,t)$, we want to find $x(t)$ over $t\in[0,2)$, given that $x(0)=1$ and $f(x,t) = 5x$. The exact solution of this equation would be $x(t) = e^{5t}$.

\begin{minted}[linenos]{python}
import numpy as np
import matplotlib.pyplot as plt
def f(x,t): # define the function f(x,t)
    return 5*x
epsilon = 0.01 # define timestep
t = np.arange(0,2,epsilon) # define an array for t
x = np.zeros(t.shape) # define an array for x
x[0]= 1 # set initial condition
for i in range(1,t.shape[0]):
    x[i] = epsilon*f(x[i-1],t[i-1])+x[i-1] # Euler Integration Step
\end{minted}

\subsubsection*{Vectorizing the Euler Method}

Euler's Method also applies to vectors and can solve simultaneous differential equations.

The Initial Value problem now becomes:

\begin{eqnarray}
	\frac{d\vec{x}}{dt} = \vec{f}(\vec{x}, t)
\end{eqnarray}
\begin{eqnarray}
	\vec{x}(t_o) = \vec{x_o}
\end{eqnarray}

where $\vec{x}=[x_1,x_2...]$ and $\vec{f}(\vec{x}, t)=[f_1(\vec{x}, t),f_2(\vec{x}, t)...]$.

The Euler's Method becomes:

\begin{eqnarray}
	t_{n+1} = t_n + \epsilon
\end{eqnarray}
\begin{eqnarray}
	\vec{x_{n+1}} = \vec{x_n} + \epsilon \vec{f}(\vec{x_n}, t_n)
\end{eqnarray}

Let $\frac{d\vec{x}}{dt}=f(\vec{x},t)$, we want to find $\vec{x}(t)$ over $t\in[0,2)$, given that $\vec{x}=[x,y]$, $\vec{x}(0)=[1,0]$ and $f(\vec{x},t) = [x-y,y-x]$.

\begin{minted}[linenos]{python}
def f(x,t): # define the function f(x,t)
    x_,y_ = x
    return np.array([x_-y_,y_-x_])
t = np.arange(0,2,epsilon) # define an array for t
x = np.zeros((2,t.shape[0])) # define an array for x
x[:,0]= [1,0] # set initial condition
for i in range(1,t.shape[0]):
    x[:,i] = epsilon*f(x[:,i-1],t[i-1])+x[:,i-1] # Euler Integration Step
\end{minted}

\subsubsection*{A Generalized function for Euler Integration}

Now, we create a generalized function that takes in 3 inputs ie. the function $f(\vec{y},t)$ when $\frac{d\vec{y}}{dt}=f(\vec{y},t)$, the time array, and initial vector $\vec{y_0}$. The Algorithm for the Generalized Function is:

\begin{itemize}
\item Get the required inputs: function $\vec{f}(\vec{x},t)$, initial condition vector $\vec{y_0}$ and time series $t$.
\item Create a zero matrix to hold the output.
\item For each timestep, perform the euler method updation with variable $\epsilon$ and store it in the output matrix.
\item Return the output timeseries matrix.
\end{itemize}

\begin{minted}[linenos]{python}
class _Integrator():
    def integrate(self,func,y0,t):
        time_delta_grid = t[1:] - t[:-1]
        y = np.zeros((y0.shape[0],t.shape[0]))
        y[:,0] = y0
        for i in range(time_delta_grid.shape[0]):
            y[:,i+1]= time_delta_grid[i]*func(y[:,i],t[i])+y[:,i]
        return y
def odeint_euler(func,y0,t):
    y0 = np.array(y0)
    t = np.array(t)
    return _Integrator().integrate(func,y0,t)
solution = odeint_euler(f,[1.,0.],t)
\end{minted}

\subsubsection*{An Introduction to TensorFlow}

TensorFlow is an open-source software library. TensorFlow was originally developed by researchers and engineers working on the Google Brain Team within Google’s Machine Intelligence research organisation to conduct machine learning and deep neural networks research, but the system is general enough to be applicable in a wide variety of other domains as well!

Essentially, TensorFlow library for high-performance numerical computation. Its flexible architecture allows easy deployment of computation across a variety of platforms (CPUs, GPUs, TPUs), and from desktops to clusters of servers. It is a python package that (much like BLAS on Intel MKL) speeds up Linear Algebra Computation. What is unique about this system is that it is capable of utilising GPUs and TPUs for computation and its written in a more straightforward language like python.

\subsubsection*{Why GPU/TPU vs CPU?}

The answer lies in the architecture: 
CPU = Faster per Core Processing, Slow but Large Memory Buffer, Few Cores
GPU/TPU = Slower Processing, Faster but Smaller Memory Buffer, Many Cores

Thus, GPUs and TPUs have been optimised for a large number of simple calculations done parallel. The extent of this parallelisation makes it suitable for vector/tensor manipulation.

\subsubsection*{Euler Integration Function in TensorFlow}

Transitioning to TensorFlow is not a trivial process, but after some practice, it is as simple as using Numpy. Because of the way the TensorFlow architecture is designed, there are a few limitations to how one can do simpler operations/manipulation. However, it is easy to overcome using the correct function and code patterns which can be easily learnt.

\begin{minted}[linenos]{python}
# Firstly, import TensorFlow
import tensorflow as tf

class _Tf_Integrator():
    def integrate(self, func, y0, t): 
        time_delta_grid = t[1:] - t[:-1]  
        # tf.scan(fn,el,init) is an iterator over elems, it 
        # applies fn recursively  on tensor init fn is function 
        # takes in two inputs: accumulated fn and the value of 
        # current iteration on el 
        y = tf.scan(scan_func, (t[:-1], time_delta_grid),y0) 
        return tf.concat([[y0], y], axis=0)
    
    # Create and return a stepper function fn
    def _make_scan_func(self, func): 
        def scan_func(y, t_dt): 
            t, dt = t_dt
            dy = dt*func(y,t)
            return y + dy
        return scan_func

def tf_odeint_euler(func, y0, t):
    # Convert input to TensorFlow Objects
    t = tf.convert_to_tensor(t, preferred_dtype=tf.float64, name='t')
    y0 = tf.convert_to_tensor(y0, name='y0')
    return _Tf_Integrator().integrate(func,y0,t)
    
# Define a function using Tensorflow math operations. 
# This creates a computational graph.
def f(X,t):
    x = X[:-1]
    y = X[1:]
    out = tf.concat([x-y,y-x],0)
    return out
y0 = tf.constant([1,0], dtype=tf.float64)
epsilon = 0.01
t = np.arange(0,2,epsilon)
# Define the final value (output of scan) that we wish to compute
state = tf_odeint_euler(f,y0,t)
# Start a TF session and evaluate state
with tf.Session() as sess:
    state = sess.run(state)
\end{minted}






























% Results and Discussion can be combined.
\section*{Results}
Nulla mi mi, venenatis sed ipsum varius, Table~\ref{table1} volutpat euismod diam. Proin rutrum vel massa non gravida. Quisque tempor sem et dignissim rutrum. Lorem ipsum dolor sit amet, consectetur adipiscing elit. Morbi at justo vitae nulla elementum commodo eu id massa. In vitae diam ac augue semper tincidunt eu ut eros. Fusce fringilla erat porttitor lectus cursus, vel sagittis arcu lobortis. Aliquam in enim semper, aliquam massa id, cursus neque. Praesent faucibus semper libero.

% Place tables after the first paragraph in which they are cited.
\begin{table}[!ht]
\begin{adjustwidth}{-2.25in}{0in} % Comment out/remove adjustwidth environment if table fits in text column.
\centering
\caption{
{\bf Table caption Nulla mi mi, venenatis sed ipsum varius, volutpat euismod diam.}}
\begin{tabular}{|l+l|l|l|l|l|l|l|}
\hline
\multicolumn{4}{|l|}{\bf Heading1} & \multicolumn{4}{|l|}{\bf Heading2}\\ \thickhline
$cell1 row1$ & cell2 row 1 & cell3 row 1 & cell4 row 1 & cell5 row 1 & cell6 row 1 & cell7 row 1 & cell8 row 1\\ \hline
$cell1 row2$ & cell2 row 2 & cell3 row 2 & cell4 row 2 & cell5 row 2 & cell6 row 2 & cell7 row 2 & cell8 row 2\\ \hline
$cell1 row3$ & cell2 row 3 & cell3 row 3 & cell4 row 3 & cell5 row 3 & cell6 row 3 & cell7 row 3 & cell8 row 3\\ \hline
\end{tabular}
\begin{flushleft} Table notes Phasellus venenatis, tortor nec vestibulum mattis, massa tortor interdum felis, nec pellentesque metus tortor nec nisl. Ut ornare mauris tellus, vel dapibus arcu suscipit sed.
\end{flushleft}
\label{table1}
\end{adjustwidth}
\end{table}


%PLOS does not support heading levels beyond the 3rd (no 4th level headings).
\subsection*{\lorem\ and \ipsum\ nunc blandit a tortor}
\subsubsection*{3rd level heading} 
Maecenas convallis mauris sit amet sem ultrices gravida. Etiam eget sapien nibh. Sed ac ipsum eget enim egestas ullamcorper nec euismod ligula. Curabitur fringilla pulvinar lectus consectetur pellentesque. Quisque augue sem, tincidunt sit amet feugiat eget, ullamcorper sed velit. Sed non aliquet felis. Lorem ipsum dolor sit amet, consectetur adipiscing elit. Mauris commodo justo ac dui pretium imperdiet. Sed suscipit iaculis mi at feugiat. 

\begin{enumerate}
	\item{react}
	\item{diffuse free particles}
	\item{increment time by dt and go to 1}
\end{enumerate}


\subsection*{Sed ac quam id nisi malesuada congue}

Nulla mi mi, venenatis sed ipsum varius, volutpat euismod diam. Proin rutrum vel massa non gravida. Quisque tempor sem et dignissim rutrum. Lorem ipsum dolor sit amet, consectetur adipiscing elit. Morbi at justo vitae nulla elementum commodo eu id massa. In vitae diam ac augue semper tincidunt eu ut eros. Fusce fringilla erat porttitor lectus cursus, vel sagittis arcu lobortis. Aliquam in enim semper, aliquam massa id, cursus neque. Praesent faucibus semper libero.

\begin{itemize}
	\item First bulleted item.
	\item Second bulleted item.
	\item Third bulleted item.
\end{itemize}

\section*{Discussion}
Nulla mi mi, venenatis sed ipsum varius, Table~\ref{table1} volutpat euismod diam. Proin rutrum vel massa non gravida. Quisque tempor sem et dignissim rutrum. Lorem ipsum dolor sit amet, consectetur adipiscing elit. Morbi at justo vitae nulla elementum commodo eu id massa. In vitae diam ac augue semper tincidunt eu ut eros. Fusce fringilla erat porttitor lectus cursus, vel sagittis arcu lobortis. Aliquam in enim semper, aliquam massa id, cursus neque. Praesent faucibus semper libero~\cite{bib3}.

\section*{Conclusion}

CO\textsubscript{2} Maecenas convallis mauris sit amet sem ultrices gravida. Etiam eget sapien nibh. Sed ac ipsum eget enim egestas ullamcorper nec euismod ligula. Curabitur fringilla pulvinar lectus consectetur pellentesque. Quisque augue sem, tincidunt sit amet feugiat eget, ullamcorper sed velit. 

Sed non aliquet felis. Lorem ipsum dolor sit amet, consectetur adipiscing elit. Mauris commodo justo ac dui pretium imperdiet. Sed suscipit iaculis mi at feugiat. Ut neque ipsum, luctus id lacus ut, laoreet scelerisque urna. Phasellus venenatis, tortor nec vestibulum mattis, massa tortor interdum felis, nec pellentesque metus tortor nec nisl. Ut ornare mauris tellus, vel dapibus arcu suscipit sed. Nam condimentum sem eget mollis euismod. Nullam dui urna, gravida venenatis dui et, tincidunt sodales ex. Nunc est dui, sodales sed mauris nec, auctor sagittis leo. Aliquam tincidunt, ex in facilisis elementum, libero lectus luctus est, non vulputate nisl augue at dolor. For more information, see \nameref{S1_Appendix}.

\section*{Supporting information}

% Include only the SI item label in the paragraph heading. Use the \nameref{label} command to cite SI items in the text.
\paragraph*{S1 Fig.}
\label{S1_Fig}
{\bf Bold the title sentence.} Add descriptive text after the title of the item (optional).

\paragraph*{S2 Fig.}
\label{S2_Fig}
{\bf Lorem ipsum.} Maecenas convallis mauris sit amet sem ultrices gravida. Etiam eget sapien nibh. Sed ac ipsum eget enim egestas ullamcorper nec euismod ligula. Curabitur fringilla pulvinar lectus consectetur pellentesque.

\paragraph*{S1 File.}
\label{S1_File}
{\bf Lorem ipsum.}  Maecenas convallis mauris sit amet sem ultrices gravida. Etiam eget sapien nibh. Sed ac ipsum eget enim egestas ullamcorper nec euismod ligula. Curabitur fringilla pulvinar lectus consectetur pellentesque.

\paragraph*{S1 Video.}
\label{S1_Video}
{\bf Lorem ipsum.}  Maecenas convallis mauris sit amet sem ultrices gravida. Etiam eget sapien nibh. Sed ac ipsum eget enim egestas ullamcorper nec euismod ligula. Curabitur fringilla pulvinar lectus consectetur pellentesque.

\paragraph*{S1 Appendix.}
\label{S1_Appendix}
{\bf Lorem ipsum.} Maecenas convallis mauris sit amet sem ultrices gravida. Etiam eget sapien nibh. Sed ac ipsum eget enim egestas ullamcorper nec euismod ligula. Curabitur fringilla pulvinar lectus consectetur pellentesque.

\paragraph*{S1 Table.}
\label{S1_Table}
{\bf Lorem ipsum.} Maecenas convallis mauris sit amet sem ultrices gravida. Etiam eget sapien nibh. Sed ac ipsum eget enim egestas ullamcorper nec euismod ligula. Curabitur fringilla pulvinar lectus consectetur pellentesque.

\section*{Acknowledgments}
Cras egestas velit mauris, eu mollis turpis pellentesque sit amet. Interdum et malesuada fames ac ante ipsum primis in faucibus. Nam id pretium nisi. Sed ac quam id nisi malesuada congue. Sed interdum aliquet augue, at pellentesque quam rhoncus vitae.

\nolinenumbers

% Either type in your references using
% \begin{thebibliography}{}
% \bibitem{}
% Text
% \end{thebibliography}
%
% or
%
% Compile your BiBTeX database using our plos2015.bst
% style file and paste the contents of your .bbl file
% here. See http://journals.plos.org/plosone/s/latex for 
% step-by-step instructions.
% 
\begin{thebibliography}{10}

\bibitem{bib1}
Conant GC, Wolfe KH.
\newblock {{T}urning a hobby into a job: how duplicated genes find new
  functions}.
\newblock Nat Rev Genet. 2008 Dec;9(12):938--950.

\bibitem{bib2}
Ohno S.
\newblock Evolution by gene duplication.
\newblock London: George Alien \& Unwin Ltd. Berlin, Heidelberg and New York:
  Springer-Verlag.; 1970.

\bibitem{bib3}
Magwire MM, Bayer F, Webster CL, Cao C, Jiggins FM.
\newblock {{S}uccessive increases in the resistance of {D}rosophila to viral
  infection through a transposon insertion followed by a {D}uplication}.
\newblock PLoS Genet. 2011 Oct;7(10):e1002337.

\end{thebibliography}



\end{document}

